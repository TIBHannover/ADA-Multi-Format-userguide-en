\documentclass{article}

\usepackage{caption}
\usepackage{graphicx}
                
\usepackage{calc}
                
\newlength{\imgwidth}
                
\newcommand\scaledgraphics[2]{%
                
\settowidth{\imgwidth}{\includegraphics{#1}}%
                
\setlength{\imgwidth}{\minof{\imgwidth}{#2\textwidth}}%
                
\includegraphics[width=\imgwidth,height=\textheight,keepaspectratio]{#1}%
                
}
            
\begin{document}

\title{Step 2: Create a Book Project in Fidus Writer}

\maketitle


The book project in Fidus Writer will act as an empty container for your publication, later on you can change all the file names and book information to reflect your books title and content. You can also add and remove documents at any time.


\subsection{What's covered here}\label{H4535132}


\begin{enumerate}
\item Create a 'personal' folder (only you see this - it's not shared) for your book documents.


\item Make three placeholder documents for your book parts: Front matter; Section 1, and; Back matter.


\item Add your documents to a Fidus Writer Book - a collation of book documents.


\item Connecting your book to a Git repo.


\end{enumerate}

In a later step, sharing the publication with your team will be covered.


Complete publication configuration details can be found in the pipeline manual.


\subsection{1. Create a 'personal' folder}\label{H4439255}



Here you will create a folder and after this create your three documents in the folder. To start with you need to be in the Document area of the website. 


At the top of the page in the secondary menu, click on 'Create New Folder' and give the folder a name.

\begin{figure}
\scaledgraphics{8c499ef1-2708-4cab-a7c0-7b428719cf92.png}{1}
\caption*{Photo 1: Document folder creation - add folder, name, and save}\label{F89073221}
\end{figure}


Now you will have an empty folder. If no documents are made in the folder, and it is left empty the folder will not be saved when you navigate away.


\subsection{2. Create placeholder documents }\label{H8214025}



We will now create three documents in the folder you have just created. These are the placeholder document examples you will make:

\begin{itemize}
\item Front Matter: Where you will add imprint, contributor information, acknowledgements, etc.


\item Section 1: A top level part of a book as section or chapter


\item Back Matter: This can contain appendices, glossaries, or abbreviations, etc.


\end{itemize}

\subsubsection{How to create documents}\label{H7006285}



In the sub-menu below Documents select 'Create new document' and choose \textbf{'Book Default'} document template. If you are working on a special book or publication series you might use a different document template. Consult the publication manager for guidance.


Here you will add three documents as placeholders. These are added, so you can configure your book basics, names and documents can be changed or deleted later. Make three documents with these name: Front Matter; Section 1, and; Back Matter.

\begin{figure}
\scaledgraphics{c08856bb-6601-4f76-9196-f85fd52b9756.png}{1}
\caption*{Photo 2: Create documents and use a document template}\label{F85303861}
\end{figure}


1. Create new document, 2. Select the Document Template, 3. Add document title, 4. Close document from the file menu. 

\begin{figure}
\scaledgraphics{ac6867d5-af90-4fcf-baf6-f48ad34c0de7.png}{1}
\caption*{Photo 3: Create document and add a title, then close in the file menu}\label{F72135581}
\end{figure}

\begin{figure}
\scaledgraphics{c6cd4404-ce9f-4c61-bbfe-cfc12b71dae7.png}{1}
\caption*{Photo 4: Adding documents to be used in your book}\label{F35194901}
\end{figure}


You now have the basic book sections, and we can move onto creating the Fidus Writer book container.


\subsection{3. Create a Fidus Writer book}\label{H642923}



A Fidus Writer Book collects together a series of Fidus Writer documents. Here we will create a book and add your docs that you have just created, as well as carry our some basic configurations of the book.


Navigate to the Book section of the website.

\begin{figure}
\scaledgraphics{d9710098-1829-44a3-b890-3fc28fe6ac91.png}{1}
\caption*{Photo 5: Create a Fidus Writer Book. Navigate to the book section, use create book on the left}\label{F94386831}
\end{figure}


Click 'Create new book'. You will be show a book dialogue box with a number of tabs: Basic information, Sections, Bibliography, Epub, Print / PDF, Validation, and Git repo.B


To start with you will only complete a few settings, you can return later to complete all of the book setup. Here we will fill out the title and add your documents.


1. Insert the book title in the Basic info tab.

\begin{figure}
\scaledgraphics{68440cbf-3463-417e-83a5-cd6a29532c0b.png}{1}
\caption*{Photo 6: Add book information, add book title to start with}\label{F90195231}
\end{figure}


2. Add documents. To add your documents move to the 'Sections' tab. Here you will see your Documents listed on the left, at the top your newly created 'folder'. Click the folder to display its contents. You can add your documents to the book by selecting them and clicking on the arrow in the middle to add them to the right column. Now save your book. The dialogue box will now close, and you will see your book listed in the Book section of the site.

\begin{figure}
\scaledgraphics{a78ac1eb-85d0-4bf2-b732-94be1691dd6a.png}{1}
\caption*{Photo 7: Select Chapters tab, add documents from your folder - left to te right column to be in your book. Then save}\label{F16811081}
\end{figure}


You can return later to complete all the book settings.


Your book is now ready to be connected to Git for outputting.


\subsection{4. Connecting your Fidus Book to a Git repo}\label{H8269040}



This part of the process only needs to be carried out by publication managers or users who will be outputting to Git. If the Git repo is public then any user will be able to see the saved content without any login credentials. Repos can be made private or access only given to specific users or groups of users.


You need to have created your Git repo in advance which is covered in Step 1., of the guide, this will repo will be where you output your publication files too.


First we will connect Fidus Writer with the Git instance you are using, this is done by authorizing Git to connect with Fidus Writer using you user accounts on both systems.


\subsubsection{Connect platforms}\label{H5660938}



1. Make sure you are logged into Git and Fidus Writer.


2. From the Fidus Writer homepage navigate to your user profile at the top right and click on your username, this will take you to your user profile page where you can connect with your Git instance in the 'social accounts area'.


3. Click Connect next to the Git instance you want to connect to.

\begin{figure}
\scaledgraphics{2254c214-1e83-4452-9c22-9ff847136c1c.png}{1}
\caption*{Photo 8: Connect to Git}\label{F41047411}
\end{figure}


4. You will now be redirected to the Git website, you will need to log in if you haven't already done so.


5. Then accept the authorization. This process connects your user accounts and allows the two systems to transfer your publication files.


The connection process is now complete, and we will now select the repo for your book.


\subsubsection{Select repo}\label{H5346176}



1. Navigate to your book and click on it to open the book dialogue box. Click on the Git repository tab on the right.

\begin{figure}
\scaledgraphics{4b4e00ad-cd99-40b1-bb1e-5689bafe807a.png}{1}
\caption*{Photo 9: Select repo to use and save. Reload if repo you want is not available}\label{F77242471}
\end{figure}


2. Click 'Refresh' on the right to get your list of repos from Git. The repos will now be available in the drop-down menu.


3. Select your repo from the list, then below the output types you need should be checked, and click save. Export format options are: EPUB export, Unpacked EPUB export, HTML export, Export Unified HTML, LaTeX export. As default, you only require EPUB and Unified HTML. PDF will be uploaded manually - instructions are in the output your book as multi-format section.


4. You can now export your book to Git.  In the Git repository tab use the export button bottom right and select 'Export to Git repository'. A dialogue will appear asking for a Commit message, this is a note for this revision export.

\begin{figure}
\scaledgraphics{3936dcb7-7576-45c8-9a91-2340d2e18fb8.png}{1}
\caption*{Photo 10: Git export settings. Git tab; select repo; choose outputs, and; export}\label{F11158451}
\end{figure}


A message dialogue will appear bottom right. When the message 'Your Book has been successfully saved to Git' appears the process has finished. 

\begin{figure}
\scaledgraphics{8f55247e-9161-49d0-8688-0cbb8299ea13.png}{1}
\caption*{Photo 11: Git export message - see bottom right-hand corner}\label{F8159591}
\end{figure}


You can now navigate to Git and you will see your files on Git. That is the end of this process.

\begin{figure}
\scaledgraphics{d80dc5a5-d385-4a3b-a03a-a752ff2686c9.png}{1}
\caption*{Photo 12: Your publication outputted to Git}\label{F54557401}
\end{figure}


\subsection{Next steps}\label{H6619707}



You can now invite your team to access the publication on Fidus Writer.

\end{document}
