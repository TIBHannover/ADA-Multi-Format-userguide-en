\documentclass{article}

\usepackage{caption}
\usepackage{tabu}
\begin{document}

\title{Step 4: Publish as Multi‑format!}

\maketitle


What is covered in the quick start guide for multi-format publishing.


For users of the system you will only need to select your layout design style from the available library of style and click output.

\begin{enumerate}
\item First the system can create many outputs from one source as 'Publication Ready Outputs' (PROs)\protect\footnotemark{}, as well as output additional interoperable and machine readable formats. 


\item The system can apply pre-made reusable templates of \textbf{'layout design styles'} with automate machine typesetting.


\item Save out styled output formats to Git at the push of a button, or preview the outputs direct from the system. Note the PDF format needs to be saved locally and then uploaded to Git (this will be automated in the near future, Sept 2022).


\end{enumerate}\addtocounter{footnote}{-1}\stepcounter{footnote}
\footnotetext{A Publication Ready Output (PRO) means that the format is ready for professional publishing, including typesetting, metadata, and other formatting and settings. Many systems can save files in a format, for example as HTML, or PDF - but it does not mean it can be used professionally. Microsoft Word can save as HTML or PDF but it doesn't make the formatted files into finished publications ready for distribution.}

\subsection{Output formats we'll cover here}\label{H6462932}



Other format outputs are listed in the System Configurations and Settings section.

\begin{itemize}
\item Website


\item Paginated Web


\item PDF


\item Print-on-demand (PDF)


\item e-Book


\end{itemize}
\begin{table}
\caption*{Table 1: Starter output formats. More formats are available but to start with we'll cover the set below.}\label{T34310601}

\begin{tabu} to \textwidth { |X|X|X|X|X|X| }
\hline



\textbf{Formats >>>} & Website & Paginated Web & PDF & Print-on-Demand (PDF) & e-Book
 \\


\textbf{Examples} & Template (to be provided) LINK & - & - & - & -
 \\


\textbf{Features} & Mobile first responsive & Fixed page & Screen PDF (symmetrical margins) & Print from one copy at a time. (recto - verso margins) & Use on e-Readers and distribute through book trade.
 \\


\textbf{Running header / footer} & Place in left menu & yes & yes &  & n /a
 \\


\textbf{Date (custom formats)} & Place in left menu & yes & yes &  & Inline
 \\


\textbf{Version (From Fidus book version No.)} & Place in left menu & yes & yes &  & Inline
 \\


\textbf{Comments} &  &  &  &  & 
 \\


\textbf{Fidus exports used to make output formats.}  &  &  &  &  & 
 \\
\hline

\end{tabu}\end{table}





\subsection{Preview outputs}\label{H5954601}



You can download any of your outputs locally from the book dialogues window.


See notes on PDF export below.


PIC 


\subsection{Applying layout design styles and Git export}\label{H2238943}



1. Navigate to the book area of the site and here click on your book to open its dialog box.


PIC book area


PIC book dialogue


2. In the book dialogue box select the tab on the right Git repository.


3. In the Git repository tab slect the following: the reposity you want to save to (this will already be selected if you used the earlier guide setup); the output formats you want to use, and then from the export button bottom right select 'Export to Git repository'.


PIC Git settings - hightligh 3 options from above


PIC Export to Git repository - selection


4. A Git dialogue will now appear called 'Commit message'. This is a note about the export you will make to Git and it will appear in the file listing for this git export. The pupose of the note is to inform other team members or Git users about your export, for example what kind of updates were made. A Commit message should be informative and you can pick your own style, noting these may be public if the Gitrepo is public.   


Click save and the export will start. The system will give you updates on the progress bottom right.


PIC Git export dialogue  'Commit message'


PIC Progress messages.


4. You can now save your book settings in the book dialogue box.


PIC book dialogue box.


5. Your export is now complete and your publication will now be on Git.


PIC Git - formats


PIC website 


From the Git export you can either have the Git content be public or private. Additionally you can manually or automatically have content distributed to other storage locations or systems.


\subsection{Exporting PDF to Git}\label{H9093471}



PDF outputs need to be saved locally and then uploaded to Git.







\end{document}
