\documentclass{article}

\usepackage{hyperref}
\usepackage{caption}
\usepackage{graphicx}
                
\usepackage{calc}
                
\newlength{\imgwidth}
                
\newcommand\scaledgraphics[2]{%
                
\settowidth{\imgwidth}{\includegraphics{#1}}%
                
\setlength{\imgwidth}{\minof{\imgwidth}{#2\textwidth}}%
                
\includegraphics[width=\imgwidth,height=\textheight,keepaspectratio]{#1}%
                
}
            
\usepackage{tabu}
\begin{document}

\title{Step 4: Publish as Multi-format !}

\maketitle


Here you will learn how to do the following: Output your publication as a website, pages website, PDF, and e-Book.

\begin{enumerate}
\item First the system can create many outputs from one source as 'Publication Ready Outputs' (PROs)\protect\footnotemark{}, as well as output additional interoperable and machine readable formats. 


\item The system can apply pre-made reusable templates of \textbf{'layout design styles'} with automate machine typesetting.


\item Save styled output formats to Git at the push of a button, or preview the outputs direct from the system. \textbf{Note:} the PDF format needs to be saved locally and then uploaded to Git (this will be automated in the near future, Sept 2022).


\end{enumerate}\addtocounter{footnote}{-1}\stepcounter{footnote}
\footnotetext{A Publication Ready Output (PRO) means that the format is ready for professional publishing, including typesetting, metadata, and other formatting and settings. Many systems can save files in a format, for example as HTML, or PDF - but it does not mean it can be used professionally. Microsoft Word can save as HTML or PDF but it doesn't make the formatted files into finished publications ready for distribution.}

\subsection{Output formats we'll cover here}\label{H6462932}


\begin{enumerate}
\item Website (responsive for mobile viewing)


\item Paginated Web (this means you have pages like a book in the browser as opposed to the default single scrolling page of a browser)


\item PDF


\item Print-on-demand (PDF)


\item e-Book


\end{enumerate}

Other format outputs are listed in the System Configurations and Settings section.

\begin{table}
\caption*{Table 1: Starter output formats. More formats are available but to start with we'll cover the set below.}\label{T34310601}

\begin{tabu} to \textwidth { |X|X|X|X|X|X| }
\hline



\textbf{Formats >>>} & 1. Website & 2. Paginated Web & 3. PDF & 4. Print-on-Demand (PDF) & 5. e-Book
 \\


\textbf{Examples} & Template (\href{https://tibhannover.github.io/ADA-Book-Template/}{Benchmark Template}) & - & - & - & -
 \\


\textbf{Features} & Mobile first responsive & Fixed page & Screen PDF (symmetrical left and right margins) & Print from one copy at a time. (recto - verso margins) & Use on e-Readers and distribute through book trade.
 \\


\textbf{Running header / footer} & Placed in left menu & yes & yes & yes & n /a
 \\


\textbf{Date (custom formats)} & Placed in left menu & yes & yes & yes & Inline
 \\


\textbf{Version (From Fidus book version No.)} & Place in left menu & yes & yes &  & Inline
 \\


\textbf{Fidus exports used to make output formats.}  & UHTML\footnote{UHTML - This stands for unified HTML. The Fidus exporter concatenates all the Document HTML files into one single HTML file.} & UHTML & PDF & PDF + Cover PDF (made separately)\footnote{Cover PDF. Covers for print-on-demand (PoD) need to be made separately at present due to different requirements made by PoD printers.} & EPUB
 \\
\hline

\end{tabu}\end{table}


\subsection{Preview outputs}\label{H5954601}



You can download any of your outputs locally from the book dialogues window. In the Export button bottom right you will a menu with the following export options:

\begin{itemize}
\item EPUB


\item HTML


\item UHTML


\item LaTeX


\item Print / PDF (Select in your browsers Print dialogue box if you want to Print or Save as PDF. Keep background graphics on, and margins set as none)


\end{itemize}
\begin{figure}
\scaledgraphics{81aa11b0-e706-4d38-9d60-97b77fdfbaf9.png}{1}
\caption*{Photo 1: Export book for preview}\label{F46160661}
\end{figure}


\subsection{Applying layout design styles and Git export}\label{H2238943}



\subsubsection{Choose multi-format style}\label{H4518313}



1. Navigate to the book area of the site and here click on your book to open its dialogue box.

\begin{figure}
\scaledgraphics{0a969aee-3f29-4e82-a3c7-1bb4f783ac91.png}{1}
\caption*{Photo 2: Select a book layout style}\label{F40472071}
\end{figure}


2. Choose your book \textbf{'layout design style' }from the 'Print / PDF' tab. As an example your can use 'Report 001' for an DIN A4 orientated layout style. Selecting a style will typeset all your outputs, and you can change style at any time, or add, and modify styles.


\subsubsection{Add an e-book cover}\label{H2562599}



For your e-book you will need to add cover artwork in the Epub tab of your book information. You can upload a image file here. The artwork can be from the cover of your PDF or from any other source. Use a JPEG file at a size of 2560 pixel x 1600 pixel or close to this. E-book platforms request different sizes, here we have used Amazon Kindle sizes as of January 2022.


\textbf{Tip:} Take the first page of your PDF output and use it as your cover. Render the PDF page 1 in a graphics program and save it as a JP'EG. For example using the open source image editor \href{https://www.gimp.org/}{GIMP} (GNU Image Manipulation Program).

\begin{figure}
\scaledgraphics{8f163c81-1787-4e57-a8c8-d70fd4930236.png}{1}
\caption*{Photo 3: Add an e-book cover}\label{F2242481}
\end{figure}


You can preview your e-book on your local machine using the open source \href{https://calibre-ebook.com/}{Calibre} e-reader.


\subsubsection{Export to Git}\label{H4535667}



\textbf{Note:} If your Git repo is public this will make your book public. Repos can be made public or private.


1. In the book dialogue box select the tab on the right Git repository.


2. In the Git repository tab select the following: the repository you want to save to (this will already be selected if you used the earlier guide setup); the output formats you want to use, and then from the export button bottom right select 'Export to Git repository'.

\begin{figure}
\scaledgraphics{3936dcb7-7576-45c8-9a91-2340d2e18fb8.png}{1}
\caption*{Photo 4: Git export settings. Git tab; select repo; choose outputs, and; export}\label{F11158451}
\end{figure}


3. A Git dialogue will now appear called 'Commit message'. This is a note about the export you will make to Git, and it will appear in the file listing for this git export. The purpose of the note is to inform other team members or Git users about your export, for example what kind of updates were made. A Commit message should be informative, and you can pick your own style, noting these may be public if the Git repo is public.


Click save, and the export will start. The system will give you updates on the progress bottom right.

\begin{figure}
\scaledgraphics{f0c15393-b51c-4a2d-acb3-09b5b3507c7e.png}{1}
\caption*{Photo 5: Add your Git 'commit message'. This is a note for others to know what was being saved to Git}\label{F93829001}
\end{figure}


4. You can now save your book settings in the book dialogue box.


5. Your export is now complete, and your publication will now be on Git.

\begin{figure}
\scaledgraphics{d80dc5a5-d385-4a3b-a03a-a752ff2686c9.png}{1}
\caption*{Photo 6: Git Repo view. After you have exported your publication you will see the files here}\label{F47008381}
\end{figure}

\begin{figure}
\scaledgraphics{8b26a358-6fdd-41c6-91aa-2a1fbbf977cc.png}{1}
\caption*{Photo 7: Git Pages. This is the website portal to your publication}\label{F47388851}
\end{figure}


From the Git export you can either have the Git content be public or private. Additionally, you can manually or automatically have content distributed to other storage locations or systems. These are both settings and configurations that are made in Git. See the full manual for these instructions.


\subsection{Exporting PDF to Git}\label{H9093471}



PDF outputs need to be saved locally and then uploaded to Git. 


Here we will create our local PDF from the browser, save it locally and then log onto Git in the browser and upload the PDF.


1. In the book dialogue box select print/pdf export from the lower right export button.

\begin{figure}
\scaledgraphics{caa95109-c881-4fc6-9688-11d00c66ecb6.png}{1}
\caption*{Photo 8: PDF export from book dialogue box}\label{F35654231}
\end{figure}


2. Now we will have your browser Print / PDF export dialogue box appear and there are some settings that need to be checked before we save the PDF file to your computer. 


a. Set output as PDF.


b. Set margin to none.


c. Have included background graphics checked as on.


Now click save and name the PDF \textbf{'}book.pdf'\textbf{.} \textbf{It is important to use the naming 'book.pdf' as Git then recognizes the PDF and adds it to the website it makes with Git Pages.} Save the file locally.

\begin{figure}
\scaledgraphics{178f3e82-a78a-4cdc-a6d6-9df4385477b6.png}{1}
\caption*{Photo 9: Print and PDF setting and save }\label{F15890551}
\end{figure}


3. Now upload the file to Git. Navigate to your repo in your browser, log into Git.

\begin{figure}
\scaledgraphics{3568f67f-6fc0-43b4-9cc5-adc109b4f9e6.png}{1}
\caption*{Photo 10: Upload your PDF to the repo}\label{F99448091}
\end{figure}


Now you are at your repo's top level view you can upload the book.pdf file. Click add file top right, select your book.pdf file, add a 'commit message', and click upload. Your book.pdf file need to be in the top level of your repo. See the screenshot below.


The process is now complete, and shortly the PDF will appear in your website top right menu.

\begin{figure}
\scaledgraphics{8b26a358-6fdd-41c6-91aa-2a1fbbf977cc.png}{1}
\caption*{Photo 11: All formats are listed top right}\label{F60935641}
\end{figure}


\subsection{Multi-format publishing configurations}\label{H6290333}



You can output as wide variety of Publication Ready Output formats as well as interoperable formats for a number of different uses, as well as the main source files from Fidus Writer as JSON files.


To read more about other formats and advanced settings see the full manual.


\subsubsection{Recommended minimum default output to Git}\label{H2167774}



Outputting a website, paginated web version, and PDF, and e-Book will be enough for readers. For this setting choose: UHTML, PDF as output types in the Git settings, and you will have all you need for these outputs.


\subsubsection{Creating print-on-demand publications}\label{H6991098}



The full process for print-on-demand (PoD) outputs is outside the scope of this guide, but here is an outline of the steps involved.


As an introduction to PoD this is a print process where you can deposit your book with a printer who will make the book available to customers worldwide on the web via book retail websites and when the customer orders a book it is printed as an individual copy locally and shipped to them. As the publisher you do not have to pay for the printing or shipping, instead this is deducted form the customer payment. As the publisher you are compensated for the sale, minus the book costs. You can also make your own bulk orders as the wholesale print cost.


\href{https://www.ingramcontent.com/publishers/print-on-demand}{Ingram services} Lightning Source and Ingram Spark are good examples of PoD services.


PoD can also be used for private publication only used internally too.


You will need an ISBN number to distribute the publication. You do not need an ISBN if you use PoD for private orders with books you do not publicly distribute.


\subsubsection{Steps to enable Print-on-demand}\label{H855085}


\begin{itemize}
\item Create an account with a PoD provider like Ingram Lightning Source for professional PoD or Ingram Spark for one-off self-publishing.


\item Make a book cover and upload your book block made in the PoD system. PoD covers need to have a front, spine, and back cover, and have a spine that vary in size depending on number of pages.


\item Set the sales price. The price can allow a surplus, or be set to break even, or even be subsidized. 


\item Publish. Your book will then go live on many retailers, and you are compensated for sales monthly. 


\end{itemize}
\end{document}
