\documentclass{article}

\usepackage{caption}
\usepackage{graphicx}
                
\usepackage{calc}
                
\newlength{\imgwidth}
                
\newcommand\scaledgraphics[2]{%
                
\settowidth{\imgwidth}{\includegraphics{#1}}%
                
\setlength{\imgwidth}{\minof{\imgwidth}{#2\textwidth}}%
                
\includegraphics[width=\imgwidth,height=\textheight,keepaspectratio]{#1}%
                
}
            
\begin{document}

\title{Step 3: Invite Your Team}

\maketitle


\textbf{This section is intended for }\emph{\textbf{publication managers}}\textbf{.}


You can invite contributors to your publication project and give them access to project documents and book.


\textbf{Note:} This setup is so that contributors can edit documents and preview the book publication as PDF, e-book, etc., without being able to export the publication to Git, or change other configurations of a book, such as: change the order of sections (chapters), or edit other book information and settings.


If your team does not yet have accounts yet then see the guide section 'What You'll Need to Get Started' to add them as users to the system.


For contributors access to a publication is a \textbf{three part process}:

\begin{enumerate}
\item First, the user has to \textbf{accept being a contact of yours}.


\item Second, you grant \textbf{document editing access}, and 


\item Third, you give \textbf{view-only access to the book} so that users can download previews.  


\end{enumerate}

Team members can also be given access for different roles, for example as reviewers or editors:

\begin{enumerate}
\item \textbf{Reviewer} with commenting only on documents, and; 


\item \textbf{Editor} with track changes only permissions on documents.


\end{enumerate}

Settings for these roles are described at the end of the section.


\subsection{1. Adding users as contacts}\label{H3777938}



Each user in Fidus Writer has contacts. First a user has to be a contact before they can be invited and given access to your documents or books.


In the homepage of Fidus Writer navigate to your user icon top right and from the drop-down menu select Contacts.

\begin{figure}
\scaledgraphics{391bdb8e-5974-4212-b315-9847119c3c0e.png}{1}
\caption*{Photo 1: Add contacts - top right}\label{F81878771}
\end{figure}


You will see an empty page if you have no contacts yet or otherwise a list of contacts.

\begin{figure}
\scaledgraphics{e5be03dd-8e38-448d-b6dd-fce0412588c5.png}{1}
\caption*{Photo 2: Invite contact - top left. List of contacts}\label{F5073131}
\end{figure}


Click \textbf{Invite contact} top left, you can add contacts here by username or email address. Each contact added will be notified about your contact request and will need to approve the request. 


If the person does not yet have a Fidus Writer account you will need to use their email address, and then they will be invited to create an account.

\begin{figure}
\scaledgraphics{50f9c291-37a8-413e-b800-f17a3b6c680d.png}{1}
\caption*{Photo 3: Invite user dialogue box. Add email address or username to invite user}\label{F73625321}
\end{figure}


The user will get a notice in Fidus Writer and as email about the contact request, and then they need to accept the request. If the user is logged in to Fidus Writer the notice will come up as a pop-up request for them to click through to contacts. Also, they can always visit their contact areas to check on your request.


You can see the status of your invite for a contact in your contacts view area. The status of an invitation is notes as \textbf{User} if the invite has been accepted.

\begin{figure}
\scaledgraphics{c21f0041-95d0-4128-abe9-970886b5b0db.png}{1}
\caption*{Photo 4: The status of an invitation is notes as User if the invite has been accepted}\label{F98742551}
\end{figure}


If you have problem adding contact then get in touch with administration support, and they can help check on the status of invites. All personal information is used in strict adherence to GDPR and principles of Digital Sovereignty where users always have to grant explicit access to their personal data.


\subsection{2. Giving users access to edit documents}\label{H9350209}



\textbf{Note:} As creator of documents you become the document owner. There can only be one document owner. Only the owner of a document can edit the sharing settings. Users that you invite will be able to \textbf{edit all parts of a document including deleting documents} as we are giving the Write access to documents. You can also set their access to being: Write tracked (track changes); Comment, or; Read (read only).


Navigate to the Fidus Writer home and the documents area and from there into the directory you made in the earlier step in the guide. Here you will see a list of your publication documents.

\begin{figure}
\scaledgraphics{ecfbd0c0-16ae-4aa4-94bc-e993d00c5c66.png}{1}
\caption*{Photo 5: Publication documents}\label{F94960221}
\end{figure}


In the directory select the top checkboxes above all the document checkboxes this will turn on and off (toggle) the selection of all the documents, then click the drop-down arrow icon and select 'Share' from the drop-menu.

\begin{figure}
\scaledgraphics{c0ffcd80-a1bb-4767-9f01-e4f6403be9db.png}{1}
\caption*{Photo 6: Select all docs by checking checkbox above documents. Note sharing drop-down is from down arrow to the right of the checkbox}\label{F83659771}
\end{figure}


You will now see the share dialogue box. Add users by moving them from the left to the right column and edit icon next to each user and change it from the view (eye icon) to edit (pencil icon) to give them full edit access, otherwise they will only be able to view documents. And then save your sharing settings. 

\begin{figure}
\scaledgraphics{2b3de32d-4d3e-4e7c-9161-6d7d37b1c232.png}{1}
\caption*{Photo 7: Select Share menu item from Drop-down document menu}\label{F25716331}
\end{figure}


The sharing task for documents is now complete.


\textbf{If you add a new user or new document - then repeat parts 1. and 2., to enable sharing.}


\subsection{3. Sharing your book for view only and preview download}\label{H6251349}



You want your contributors to be able to view the book settings and preview the complete book in its different typeset layout formats, but prevent them from publishing the book or directly rearranging book sections or selecting a new layout typesetting style, etc.


In this part we will share the book with the same users as before in documents, but with the \textbf{permissions as view only}.


1. Navigate to the book site area of Fidus Writer and locate your book.

\begin{figure}
\scaledgraphics{a0f1beec-79fa-4ed4-bc58-9c22e9536863.png}{1}
\caption*{Photo 8: Book site area}\label{F66888411}
\end{figure}


2. To the right of your book click the pencil icon. This will bring up the sharing dialogue box. As before with sharing documents move the users from the left column to the right column to share the document with them. The difference this time is that we're going to leave the users as \textbf{view only (eye icon)}.


Once you have completed this part the sharing setup in total is now completed.

\begin{figure}
\scaledgraphics{b41a1892-6720-486f-8b11-e06d46aa69ec.png}{1}
\caption*{Photo 9: Share your publication. Edit pencil icon to right of book, then in book dialogue box more users to right column and set to edit (pencil icon)}\label{F29566181}
\end{figure}


\subsection{Adding reviewer and editors to documents}\label{H5018454}



For documents, you have the option to set a users access rights as view only, comment only, or as track changes only.


These setting are useful for reviewers and editors.

\begin{figure}
\scaledgraphics{c9b1abcb-0015-4636-b6e0-d679a5618852.png}{1}
\caption*{Photo 10: Document sharing options for contributors and reviewers (Basic: write, write tracked, comments, read. Review: no comments, review, review tracked.)}\label{F15312041}
\end{figure}


\subsubsection{Basic (contributors)}\label{H7032777}


\begin{itemize}
\item Write


\item Write tracked


\item Comment


\item Read


\subsubsection{Review}\label{H2377842}



\item No comments


\item Review


\item Review tracked


\end{itemize}

\subsection{Next steps}\label{H4422568}



Next we will look at outputting your publication to Git. This will be the fourth and final step in this guide for your publication workflow.

\end{document}
