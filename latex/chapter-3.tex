\documentclass{article}

\begin{document}

\title{Pipeline Advantages}

\maketitle


The pipeline is about automation of publishing and connects the word processor to directly publish using single source publishing technology and methods.


\textbf{Single Source Publishing:} Edit in one place and distribute as multi-format to different locations automatically to produce professional publication-ready-outputs.

\begin{itemize}
\item No more sharing files over email


\item No more confusion over which is the correct document version


\item No more delays waiting for edits and reviews


\item No more lengthy delays waiting to get back layouts designs


\item No more complexity of multi-format publishing tracking down corrections and edits


\item No more delays in distribution to internal or external channels


\end{itemize}

\textbf{Publish as multi-format:} PDF; print-on-demand (PoD); web (mobile first); paged web; e-book; interoperable formats - JATS, DOCX, HTML, EPUB, LaTeX, JSON, etc.


\textbf{Instant automated typesetting and layout design:} Reusable templated layouts to typeset multi-format outputs at the press of a button. This means no time-consuming delay for layout designs during production. Instead, the layout designer can make the design templates in advance.


\textbf{Preview multi-format layouts:} Contributors to the publication can be given access to preview the different multi-format outputs.


\textbf{Co-creation:} An online real-time word processor is used, which allows multiple users to work together at the same time. This means authors, editors, reviewers, and designers can all work on one document at the same time.


\textbf{High quality academic word processor:} Features include access to citation databases like EuropePMC, footnotes and citations, citation styles, optional figures and table captioning and lists.


\textbf{Versioning:} Versioned storage is used where edition releases can be made, with all earlier versions being available. Additionally, all edits are stored, so changes can be tracked, audited, and if needed to be reversed. Versioning is recorded with cryptographic IDs for precise editing and validation of document versions.


\textbf{Automatic distribution:} Multi-format outputs can be distributed to internal organization or external locations and channels. 


\textbf{Automatic website creation:} Websites can be created automatically for publications and these can be made to be public or private.


\textbf{GitLab Infrastructure:} GitLab Community Edition is used as a self-hosted option for storage of publications. GitLab provides a powerful infrastructure for content processing, distribution, automated tasks, and teamwork.


\textbf{Semantic by design:} From the start of editing the publication content is semantically structured – which is the key to allowing for automation. Additional layers of semantic structuring can be added for adding meaning to document structures, to using Linked Open Data, ontologies, and controlled vocabularies to structure the meaning of the publications content.


\textbf{Enhanced publications:} Modern Open Science publishing offers a number of publication feature enhancements that can be deployed. Persistent identifiers (PIDs) to correctly identify organizations (ROR), persons (ORCID), and documents (DOI). Open licencing ensures reuse if needed. Interoperable formats ensure reuse, discoverability, and portability. Machine readable content and metadata ensures content is FAIR compliant (Findability, Accessibility, Interoperability, and Reusable). Linked Open Data markup to ensure content is structured and reusable in knowledge systems and AI/ML.


\textbf{Digital sovereignty by design:} All aspects of the system take personal data and privacy, and data security into account. These measures include; GDPR compliance; using open source software for code auditing; using secure self-hosting from on-site hosting to designated jurisdiction cloud hosting; as well as secure DevOps procedures.

\end{document}
