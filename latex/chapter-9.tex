\documentclass{article}

\begin{document}

\title{Glossary}

\maketitle


Terms used in the publishing pipeline.


\textbf{Automatic typesetting} – The use of heuristic machine rules to typeset a publication.


\textbf{Creative Commons Licence} – A Creative Commons (CC) license is one of several public copyright licenses that enable the free distribution of an otherwise copyrighted "work".


\textbf{Git cryptographic ID} – A way of giving a unique identifier using Git Commit ID (SHA) to content stored using Git.


\textbf{Digital Sovereignty} – Digital Sovereignty is the degree of control an individual, organization or government has over the data they generate and work with on local or online platforms.


\textbf{DOI} – A DOI (Digital Object Identifier) is a unique and never-changing string assigned to online publication and their subcomponents - chapters, images, videos, etc. 


\textbf{FAIR / FAIR Data} – FAIR data are data which meet principles of findability, accessibility, interoperability, and reusability (FAIR).


\textbf{Fidus Writer} – Fidus Writer is an online collaborative editor especially made for academics who need to use citations and/or formulas.


\textbf{Git} – Git is a free and open source 'distributed version control system'. A distributed version control is a form of version control (management of changes) in which the complete codebase, including its full history, is mirrored on every developer's computer.


\textbf{GitHub} – GitHub is an Internet hosting service for software development and version control using Git.


\textbf{GitHub Pages} – GitHub Pages is a static web hosting service.


\textbf{GitLab} – GitLab is an open source Internet hosting service for software development and version control using Git.


\textbf{GitLab Pages} – GitLab Pages is a static web hosting service to publish from a repository in GitLab.


\textbf{Linked Open Data} – Linked Open Data is a set of design principles for sharing open machine-readable interlinked data on the Web. 


\textbf{Multiformat publishing} – Publishing as formats such as print, PDF, web, e-book. Multi-format publishing has to take into account the limitations of each format, e.g., can the format support tables, video, or hyperlinks. Other considerations are related to navigation and presentation, e.g., formats like the web tend not to be paginated which alters the use of citation of a printed page number. Lastly each format has specific metadata considerations and distribution channels.


\textbf{Open access} – Open Access is a convention in academic publishing to make publications freely accessible.


\textbf{Open data} – Open data is data that is openly accessible, exploitable, editable and shared by anyone for any purpose, even commercially. Open data is licensed under an open license.


\textbf{Open science} – Open science is the movement to make scientific research (including publications, data, physical samples, and software) and its dissemination accessible to all levels of society, amateur or professional.


\textbf{Open source software} – Open-source software (OSS) is computer software that is released under a license in which the copyright holder grants users the rights to use, study, change, and distribute the software and its source code to anyone and for any purpose.


\textbf{ORCID} – ORCID (Open Researcher and Contributor ID) is a nonproprietary alphanumeric code to uniquely identify authors and contributors of scholarly communication. ORCID is a persistent identifier. 


\textbf{Paginated web (Paged Web)} – Paginated web is the presentation of web pages as sequences of pages in the form of a codex or book. 


\textbf{Persistent identifier (PID)} – A persistent identifier is a long-lasting reference to a document, file, web page, or other object.


\textbf{Publication Ready Outputs} – A Publication Ready Output (PRO) means that the format is ready for professional publishing, including typesetting, metadata, and other formatting and settings. Many systems can save files in a format, for example as HTML, or PDF - but it does not mean it can be used professionally. Microsoft Word can save as HTML or PDF, but it doesn't make the formatted files into finished publications ready for distribution.


\textbf{Repository / Repo (Git Repo)} – Repositories in GIT contain a collection of files of various different versions of a Project. These files are imported from the repository into the local server of the user for further updates and modifications in the content of the file. A VCS or the Version Control System is used to create these versions and store them in a specific place termed a repository.


\textbf{ROR} – ROR is a community-led project to develop an open, sustainable, usable, and unique identifier for every research organization in the world. ROR is a persistent identifier (PID). 


\textbf{Single Source Publishing} – Single-source publishing is a content management method which allows the same source content to be used across different forms of media and more than one time.


\textbf{Versioning (Git)} – Version control, the management of changes to documents, computer programs, large websites, and other collections of information


\textbf{Vivliostyle (CSS Typesetting)} – An open source project for a new typesetting system fitting for digital and web publishing based on the latest web standard technology.

\end{document}
